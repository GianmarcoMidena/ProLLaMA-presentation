\begin{frame}[shrink=10]{Evaluation Metrics - Protein Generation}
	\begin{columns}
		\begin{column}{0.5\textwidth}
			\begin{itemize}\setlength\itemsep{1em}
				\item structural plausibility of a protein
				\begin{itemize}
					\item \textbf{pLDDT}: Local Distance Difference Test~(\cite{jumper2021highly})
					\begin{itemize}
						\item \red{Unreliable with Intrinsically Disordered Regions (IDRs)}
						\item Tool: OmegaFold
					\end{itemize}
					\item \textbf{SC-Perp}: Self-Consistency Perplexity~(\cite{alamdari2023protein})
					\begin{itemize}
						\item Tool: OmegaFold, ProteinMPNN
					\end{itemize}
				\end{itemize}
				\item structural similarity between generated and known proteins
				\begin{itemize}
					\item \textbf{TM-score}: Template Modeling score~(\cite{zhang2004scoring})
					\item \textbf{RMSD}:  Root-Mean-Square Deviation
					\begin{itemize}
						\item Atomic distance
					\end{itemize}
					\item Tool: Foldseek
					\item Reference Protein Databases: AFDB, PDB
				\end{itemize}
			\end{itemize}
		\end{column}
		\begin{column}{0.5\textwidth}
			\begin{itemize}\setlength\itemsep{1em}
				\item sequence similarity between generated and known proteins
				\begin{itemize}
					\item \textbf{Seq-Ident}: Sequence Identity
					\item Low = sequence diversity
					\item Tool: Foldseek
					\item Reference Protein Databases: AFDB, PDB
				\end{itemize}
				\item homology between generated and known proteins
				\begin{itemize}
					\item \textbf{H-Prob}: Homologous Probability
					\begin{itemize}\setlength\itemsep{1em}
						\item Probability that a generated protein is homologous to a known one
					\end{itemize}
					\item Homologous proteins have a common evolutionary origin, shared ancestry
					\item Tool: Foldseek
					\item Reference Protein Databases: AFDB, PDB
				\end{itemize}
			\end{itemize}
		\end{column}
	\end{columns}
	\vspace{1em}
	%What does homology mean?
	Note: since averages are considered, there are instances where generated proteins outperform natural proteins.
\end{frame}